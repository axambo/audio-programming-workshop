\documentclass[screen, aspectratio=43]{beamer}
\usepackage[T1]{fontenc}
\usepackage[utf8]{inputenc}

% Use the NTNU-temaet for beamer 
% \usetheme[style=ntnu|simple|vertical|horizontal, 
%     language=bm|nn|en, 
%     smalltitle, 
%     city=all|trondheim|alesund|gjovik]{ntnu2017}
\usetheme[style=ntnu,language=en]{ntnu2017}

\usepackage[english]{babel}
\usepackage[style=numeric,backend=biber,natbib=false,sorting=none]{biblatex}

\title[AP-intro]{MCT4048: Audio Programming}
\subtitle{The Extensions: Mobile Music}
\author[A. Xamb{\'o}]{Anna Xamb{\'o}}
\institute[NTNU]{Department of Music, NTNU}
\date{14 February 2019}
%\date{} % To have an empty date

\addbibresource{../ap.bib} % Add bibliography database

% Set the reference style to numeric.
% See here: http://tex.stackexchange.com/questions/68080/beamer-bibliography-icon
\setbeamertemplate{bibliography item}[text] 

% Set bibliography fonts to a small size.
\renewcommand*{\bibfont}{\footnotesize}

\begin{document}

\begin{frame}
  \titlepage
\end{frame}

% Alternatively, special title page command to get a different background
% \ntnutitlepage
%
\begin{frame}
\frametitle{Start setting up...}
Download: \url{https://github.com/axambo/audio-programming-workshop/} 
\\
\vspace{10 mm}
Go to: \textrm{code/d7/00-setting-up/checklist.md}
\end{frame}
%
\begin{frame}
\frametitle{This Week: The Extensions (40\% Group Work)}
\begin{itemize}
\item \textbf{Syllabus}: \url{https://uio.instructure.com/courses/17406}
\item \textbf{Assignment 4} (Total grade: 30\%): Presentation mini-project 2 (group) -- \textcolor{olive}{days 6 (February 13, 2019) (10\%)},  \textbf{\textcolor{olive}{7 (February 14, 2019) (10\%)}}, 8 (February 15, 2019) (10\%)
\item \textbf{Assignment 5} (Total grade: 10\%):  Written blog post about the mini-project 2 (group) -- February 22, 2019
\end{itemize}
\end{frame}
%
\begin{frame}
\frametitle{Assignment 4 Part 3 (10\%) - Final Presentation Mini-Project}
%The order of the groups will be provided by the shuffle algorithm :)
Each group will have 15 minutes (only questions within this time) to explain the following using slides and live demo:
\begin{itemize}
\item \textbf{Description}: What is the title of the project and main concept. Overview of the technologies used.
\item \textbf{Timeline}: Provide an overview of the 3 days that you have been working in the mini-project. What have you been working on?
\item \textbf{Division of labour}: Explain who has been working in what, how you have documented it, working strategies and technologies used.
\item  \textbf{Live demo}: Try to allocate some time for a live demo.
\item \textbf{Achievements}: Give a summary of the achievements of this week through the project. Any progress?
\item \textbf{Challenges}: Give a summary of the challenges that you have been encountering over the week and how you did face them?
\item \textbf{What is next?} e.g. blog post, code repository, website publishing... more development?
\end{itemize}
\end{frame}
%
\begin{frame}
\frametitle{Assignment 5 (10\%) - Written Blog Post Mini-Project 2}
\textbf{Deadline}: February 22, 2019 at 5pm
\begin{itemize}
\item Make sure to include the information requested in your presentation (see previous slide) plus any additional feedback received or follow-up insights.
\item Be familiar with the new metadata to be used so that the blog post has author name, date, the right category (``Audio-Programming''), and a thumbnail for the homepage. See the documentation at: \url{https://github.com/MCT-master/mct-master.github.io/blob/master/README.md}.
\item At least you should use a screenshot and an embedded video showcasing the mini-project. It is a plus if you add links to the source code and a live demo link (e,g. directory in a personal webpage).
\item Add relevant links in the text to generate traffic.
\end{itemize}
\end{frame}
%
\begin{frame}
\frametitle{Program: Day 7 -- 14 February, 2019}
\begin{itemize}
\item 9.15-9.30: Setting up computers with the tools for the tutorial
\item 9.30-10.30: Tutorial: Mobile Music
\item 10:30-12:30: Mini-project 2 development (2/4)
\item 12.30-13.00: Lunch break
\item 13:00-15:00: Mini-project 2 development (2/4)
\item 15.00-16.00: Speedy presentations mini-project 2 (2/3)
\end{itemize}
\end{frame}
%
\begin{frame}
\frametitle{Learning Outcomes}
\begin{itemize}
\item Get a sense of responsive design applied to Web Audio apps.
\item Get familiar with the Bootstrap framework.
\item Be able to work in a group project relating audio programming concepts and building up from previous knowledge.
\item Be aware of best practices in web development in group projects.
\end{itemize}
\end{frame}
%
\begin{frame}
\frametitle{Responsive Design -- Principles}
\begin{itemize}
\item The use of HTML/CSS to automatically make it look good on multiple platforms and devices (desktops, tablets, phones).
\item The elements of the website are automatically resized, hide, shrink/enlarged to deliver a suitable experience.
\item A webpage should look good on any device!
\vspace{10 mm}
\center{\tiny{\url{HTML Responsive Web Design: https://www.w3schools.com/html/html_responsive.asp}}}
\end{itemize}
\end{frame}
%
\begin{frame}
\frametitle{Responsive Design -- Frameworks}
There are a number of Frameworks that offer Responsive Design.
\begin{itemize}
\item W3.CSS uses a responsive style sheet: \url{https://www.w3schools.com/w3css/}
\item Bootstrap uses HTML, CSS and jQuery to make responsive web pages: \url{https://getbootstrap.com/}
\end{itemize}
\end{frame}
%
\begin{frame}
\frametitle{Touch Events}
\begin{itemize}
\item Touch events are supported by Chrome and Firefox on desktop, and by Safari on iOS and Chrome and the Android browser on Android, as well as other mobile browsers like the Blackberry browser.
\item There are three basic touch events: 
\begin{itemize}
\item \texttt{touchstart} (a finger is placed on a DOM element), 
\item \texttt{touchmove} (a finger is dragged along a DOM element) and 
\item \texttt{touchend} (a finger is removed from a DOM element).
\end{itemize}
\item Your applications should support both touch and mouse.
\vspace{10 mm}
\center{\tiny{Touch and Mouse: \url{https://www.html5rocks.com/en/mobile/touchandmouse/}}}
\end{itemize}
\end{frame}
%
\begin{frame}
\frametitle{Mini-project development (3/4)}
You are expected to create a mini-project in teams that should be doable within a week. The overall aim is to explore a little bit further Web Audio. Here are different approaches that you can take:
\begin{itemize}
\item Develop an idea based on what we are seeing in class. Feel free to build up everyday, or change if not convinced (from scratch approach).
\item Adapt an existing code to your needs and document what are the changes (remake approach).
\item Combine projects from last week (hybrid approach).
\item Other?
\end{itemize}
\end{frame}
%
\begin{frame}
\frametitle{Working style}
\begin{itemize}
\item Work with the same team throughout the week, ideally across campuses. 
\item Make sure to clarify who has developed what part of the code. For example, divide the work into functions and add the author name at the header of each function.
\item The instructors in both sites will keep an eye on the groups to catch up.
\item There will be 4 time slots during the week to work on the project. 
\item Keep a research journal.
\end{itemize}
\end{frame}
%
\begin{frame}
\frametitle{Relevant Links}
\begin{itemize}
\item Syllabus: \url{https://uio.instructure.com/courses/17406/pages/syllabus}
\item GitHub slides \& code: \url{https://github.com/axambo/audio-programming-workshop}
\end{itemize}
\end{frame}
%
%\begin{frame}
%  \frametitle{References}
%  \printbibliography
%\end{frame}
%
\end{document}
