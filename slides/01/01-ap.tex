\documentclass[screen, aspectratio=43]{beamer}
\usepackage[T1]{fontenc}
\usepackage[utf8]{inputenc}

% Use the NTNU-temaet for beamer 
% \usetheme[style=ntnu|simple|vertical|horizontal, 
%     language=bm|nn|en, 
%     smalltitle, 
%     city=all|trondheim|alesund|gjovik]{ntnu2017}
\usetheme[style=ntnu,language=en]{ntnu2017}

\usepackage[english]{babel}
\usepackage[style=numeric,backend=biber,natbib=false,sorting=none]{biblatex}

\title[AP-intro]{MCT4048: Audio Programming}
\subtitle{The Fundamentals: Playing Sounds}
\author[A. Xamb{\'o}]{Anna Xamb{\'o}}
\institute[NTNU]{Department of Music, NTNU}
\date{5 February 2019}
%\date{} % To have an empty date

\addbibresource{../ap.bib} % Add bibliography database

% Set the reference style to numeric.
% See here: http://tex.stackexchange.com/questions/68080/beamer-bibliography-icon
\setbeamertemplate{bibliography item}[text] 

% Set bibliography fonts to a small size.
\renewcommand*{\bibfont}{\footnotesize}

\begin{document}

\begin{frame}
  \titlepage
\end{frame}

% Alternatively, special title page command to get a different background
% \ntnutitlepage

\begin{frame}
\frametitle{Survey}
\url{https://goo.gl/C1gkae}
\end{frame}

\begin{frame}
\frametitle{Start setting up...}
\url{https://github.com/axambo/audio-programming-workshop/} 
\\
Go to: code / d1 / 00-setting-up / checklist.md

\end{frame}

\begin{frame}
\frametitle{Warm-up Activity}
Mind map exercise: What is Web Audio? Pros and Cons?
\end{frame}

\begin{frame}
\frametitle{Mind Map}
\end{frame}

\begin{frame}
\frametitle{Pros}
\begin{itemize}
\item Easy access.
\item Broad distribution.
\item Social features.
\item Interactivity / audiovisual / media-rich content.
\item Internet-based.
\item Cross-platform.
\item Fast development.
\item Multiple online resources available.
\end{itemize}
\end{frame}

\begin{frame}
\frametitle{Cons}
\begin{itemize}
\item Computational performance is limited (code execution is slower than compiled programming languages (Java, C++)).
\item Consider writing a native application if your program needs to execute computationally intensive algorithms.
\item Web browsers designed to favor user experience, single-thread computations (AudioWorklet is a workaround).
\item Dependent on Internet connectivity.
\item Ephemerality: sustainability, maintenance and variability of platforms (form factor, input method, computational power...).
\end{itemize}
\end{frame}

\begin{frame}
\frametitle{This Week: The Fundamentals (40\% Individual Work)}
\begin{itemize}
\item Syllabus: \url{https://uio.instructure.com/courses/17406}
\item Assignment 1 (Total grade: 10\%): Presentation WAC paper (individual) -- day 3 (February 7, 2019) or 4 (February 8, 2019)
\item Assignment 2 (Total grade: 20\%): Presentation mini-project 1 (individual) -- days 2 (February 6, 2019) (5\%), 3 (February 7, 2019) (5\%), 4 (February 8, 2019) (10\%)
\item Assignment 3 (Total grade: 10\%): Written blog post about the mini-project 1 -- February 11, 2019
\end{itemize}
\end{frame}

\begin{frame}
\frametitle{Program: Day 1 -- 5 February, 2019}
\begin{itemize}
\item 9.15-10.00: Setting up computers with the tools for the tutorial
\item 10.00-12.30: Tutorial: Playing sounds
\item 12.30-16.00: Mini-project 1 development (1/4)
\end{itemize}
\end{frame}

\begin{frame}
\frametitle{Learning Outcomes}
\begin{itemize}
\item Understand the pros and cons of using Web Audio for audio programming.
\item Get familiar with a toolset of web technologies to start developing programs for the web based on audio.
\item Be able to find suitable information from the Web Audio API and related webpages / projects and adapt it to own needs.
\end{itemize}
\end{frame}

\begin{frame}
\frametitle{Setting Up...}
Follow the instructions from the ``00-setting-up'' folder.
\end{frame}

\begin{frame}
\frametitle{DOM}
The Document Object Model, usually referred to as the DOM, is an essential part of making websites interactive. It is an interface that allows a programming language to manipulate the content, structure, and style of a website. JavaScript is the client-side scripting language that connects to the DOM in an internet browser.
\center{\tiny{\url{https://www.digitalocean.com/community/tutorials/introduction-to-the-dom}}}
\end{frame}

\begin{frame}
\frametitle{Tutorial}
\url{https://github.com/axambo/audio-programming-workshop/tree/master/code/d1}
\end{frame}

\begin{frame}
\frametitle{Connecting audio nodes}
\begin{itemize}
\item LittleBits: \url{https://www.youtube.com/watch?v=4th8p0jSK9E}
\item PureData: \url{https://www.rebeltech.org/2016/04/07/pure-data-patch-introduction/}
\item WebAudio: \url{https://developer.mozilla.org/en-US/docs/Web/API/Web_Audio_API/Basic_concepts_behind_Web_Audio_API}
\end{itemize}
\end{frame}

\begin{frame}
\frametitle{The Web Audio API}
\begin{itemize}
\item The Web Audio API involves handling audio operations inside an audio context, and has been designed to allow modular routing.
\item Basic audio operations are performed with audio nodes, which are linked together to form an audio routing graph.
\end{itemize}
\center{\tiny{\url{https://developer.mozilla.org/en-US/docs/Web/API/Web_Audio_API}}}
\end{frame}

\begin{frame}
\frametitle{OscillatorNode}
The OscillatorNode interface represents a periodic waveform, such as a sine or triangle wave. It is an AudioNode audio-processing module that causes a given frequency of wave to be created.
\center{\tiny{\url{https://developer.mozilla.org/en-US/docs/Web/API/OscillatorNode}}}
\end{frame}

\begin{frame}
\frametitle{GainNode}
The GainNode interface represents a change in volume. It is an AudioNode audio-processing module that causes a given gain to be applied to the input data before its propagation to the output.
\center{\tiny{\url{https://developer.mozilla.org/en-US/docs/Web/API/GainNode}}}
\end{frame}

\begin{frame}
\frametitle{AudioBufferSourceNode}
The AudioBufferSourceNode represents an audio source consisting of in-memory audio data, stored in an AudioBuffer. It's especially useful for playing back audio which has particularly stringent timing accuracy requirements, such as for sounds that must match a specific rhythm and can be kept in memory rather than being played from disk or the network. 
\center{\tiny{\url{https://developer.mozilla.org/en-US/docs/Web/API/AudioBufferSourceNode}}}
\end{frame}

\begin{frame}
\frametitle{Mini-project development}
You are expected to create a mini-project that should be doable within a week. The overall aim is to get familiar with web audio. Here are different approaches that you can take:
\begin{itemize}
\item Develop an idea based on what we are seeing in class.
\item Adapt an existing code to your needs and document what are the changes.
\item Other?
\end{itemize}
\end{frame}

\begin{frame}
\frametitle{Working style}
\begin{itemize}
\item Individual work but in shared rooms. You are encourage to share and discuss with your peers.
\item One-to-one talks via Zoom or personally with the instructor to catch up.
\item There will be 4 time slots during the week to work on the project. It is okay to change the topic over the course of the week. Keep a research journal.
\end{itemize}
\end{frame}

%
\begin{frame}
\frametitle{Relevant Links}
\begin{itemize}
\item Syllabus: \url{https://uio.instructure.com/courses/17406/pages/syllabus}
\item GitHub slides \& code: \url{https://github.com/axambo/audio-programming-workshop}
\end{itemize}
\end{frame}
%
%\begin{frame}
%  \frametitle{References}
%  \printbibliography
%\end{frame}
%
\end{document}
